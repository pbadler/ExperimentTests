%---------------------------------------------
% This document is for pdflatex
%---------------------------------------------
\documentclass[11pt]{article}

\usepackage{amsmath,amsfonts,amssymb,graphicx,setspace,authblk}
\usepackage{float}
\usepackage[running]{lineno}
\usepackage[vmargin=1in,hmargin=1in]{geometry}
\usepackage{caption}

%%%  Sorting references by year(?)
\usepackage[authoryear]{natbib}
%%%

\graphicspath{ {figures/} }


\usepackage{enumitem}
\setlist{topsep=.125em,itemsep=-0.15em,leftmargin=0.75cm}

\usepackage{gensymb}

\usepackage[compact]{titlesec} 

\usepackage{bm,mathrsfs}

\usepackage{ifpdf}
\ifpdf
\DeclareGraphicsExtensions{.pdf,.png,.jpg}
\usepackage{epstopdf}
\else
\DeclareGraphicsExtensions{.eps}
\fi

\renewcommand{\floatpagefraction}{0.98}
\renewcommand{\topfraction}{0.99}
\renewcommand{\textfraction}{0.05}

\clubpenalty = 10000
\widowpenalty = 10000

%%%%%%%%%%%%%%%%%%%%%%%%%%%%%%%%%%%%%%%%%%%%% 
%%% Just for commenting
%%%%%%%%%%%%%%%%%%%%%%%%%%%%%%%%%%%%%%%%%%%%
\usepackage[usenames]{color}
\newcommand{\new}{\textcolor{red}}
\newcommand{\spe}{\textcolor{blue}}
\newcommand{\comment}{\textcolor{black}}

\newcommand{\be}{\begin{equation}}
\newcommand{\ee}{\end{equation}}
\newcommand{\ba}{\begin{equation} \begin{aligned}}
\newcommand{\ea}{\end{aligned} \end{equation}}

\def\X{\mathbf{X}}

\floatstyle{boxed}
\newfloat{Box}{tbph}{box}


\linespread{1.15}

\begin{document}

\setcounter{page}{1}
\setcounter{equation}{0}
\setcounter{figure}{0}
\setcounter{section}{0}
\setcounter{table}{0}

\centerline{\Large \textbf{Appendix S1}}

\vspace{0.2in} 

\noindent \textbf{Adler et al., ``Weak interspecific interactions in a sagebrush steppe? Conflicting evidence from observations and experiments,'' \textit{Ecology}. }

\vspace{0.2in} 

\renewcommand{\theequation}{S\arabic{equation}}
\renewcommand{\thetable}{S\arabic{table}}
\renewcommand{\thefigure}{S\arabic{figure}}
\renewcommand{\thesection}{\arabic{section}}

\linespread{1.75}


\section{Detailed Methods} 
\label{AppendixS1}

\subsection{Statistical models of vital rates}

We modeled the survival probability, $S$, of genet $i$ in species $j$ (which is in quadrat group $g$ and removal treatment $h$), 
from time $t$ to $t+1$  as
\begin{equation}
\mbox{logit}(S_{ij,t+1}) = \gamma_{j,t}^S + \varphi_{jg}^S+  \chi_{jh}^S  + \beta_{j,t}^S u_{ij,t} +  
\sum \limits_{m} \omega_{jm}^S {W}_{ijm,t}
\label{eqn:survReg}
\end{equation}
where $\gamma$ is a time-dependent intercept (year effect), $\varphi$ is the coefficient for the 
effect of spatial location group, $\chi$ is the removal treatment effect,and  $\beta$ is the coefficient that 
represents the effect of log genet size, $u$, on survival. 
$W_{ijm,t}$ is the crowding of individual $i$ in species $j$ by neighbors of species $m$ (explained below), and 
$\omega_{jm}$ is an interaction coefficient which determines the impact of crowding by species $m$ on the focal species $j$. 

Our growth model has a similar structure. The change in genet size from time $t$ to $t+1$ , conditional on survival, is given by:
\begin{equation}
u_{ij,t+1} = \gamma_{j,t}^G + \varphi_{jg}^G+  \chi_{jh}^G  + \beta_{j,t}^G u_{ij,t} + 
\sum \limits_{m} \omega_{jm}^G {W}_{ijm,t} + \varepsilon_{ij,t}^G .
\label{eqn:growReg}
\end{equation}
To capture the non-constant error variance in growth, we modeled the variance $\varepsilon$ about the growth 
curve (\ref{eqn:growReg}) as a nonlinear function of predicted genet size:
\begin{equation}
Var(\varepsilon_{ij,t}^G) = a \exp(b\hat{u}_{ij,t+1}).
\label{eqn:growVar}
\end{equation}
where $\hat{u}_{ij,t+1}$ is the right-hand side of (\ref{eqn:growReg}) without $\varepsilon$. 

Our model includes crowding from the four dominant species, \textit{A. tripartita}, \textit{H. comata}, \textit{P. spicata}, and \textit{P. secunda}, as well as two covariates representing total crowding from (1) all other perennial grasses and shrubs, which were mapped as polygons, and (2) all forb species, which were mapped as points. The next section (\textit{Modeling local crowding}) provides a detailed explanation of how we modeled crowding based on the size of and distance to neighboring plants around each focal genet.

We model recruitment at the quadrat level rather than at the individual genet level because the mapped data do not allow 
us to determine which recruits were produced by which potential parent genets. We assume that the number of individuals, $y$, of species $j$ recruiting at 
time $t+1$ in the location $q$ follows a negative binomial distribution:
\begin{equation}
y_{jq,t+1}= NegBin(\lambda_{jq,t+1},\theta) 	   
\label{eqn:recrDataModel}
\end{equation}
where $\lambda$ is the mean and $\theta$ is the size parameter (\texttt{dnegbin(p,$\theta$)} in the BUGS language, with $p=\theta/(\theta+\lambda)$). 
In turn, $\lambda$ depends on the composition of the quadrat in the previous year:
\begin{equation}
\lambda_{jq,t+1} = C'_{jq,t} \exp{\left(\gamma_{j,t}^R +  \varphi_{jg}^R + \chi_{jh}^R + 
\sum \limits_{m} \omega_{jm}^R \sqrt{C'_{mq,t}} \right)}
\label{eqn:recrProcessModel}
\end{equation}
where the superscript $R$ refers to Recruitment, $C'_{jq,t}$ is the `effective cover' (cm$^2$) of species $j$ in quadrat $q$ at time $t$, $\gamma$ 
is a time-dependent intercept, $\varphi$ is a coefficient for the effect of spatial location group,
$\chi$ is the removal treatment effect, and $\omega_{jm}$ is the coefficient that determines the effect of species $m$ cover on recruitment by species $j$. 
Eqn (\ref{eqn:recrProcessModel}) is essentially the Ricker model for discrete time population growth \citep{ricker_stock_1954}, except that we use the square root of cover in the density-dependent term. We previously used a Ricker equation \citep{adler_coexistence_2010}, but using the square-root transformation instead gave a lower Deviance Information Criterion (DIC) value \citep{spiegelhalter_bayesian_2002} for the recruitment model. Following previous work \citep{adler_coexistence_2010}, we treated year and spatial group variables as random factors, allowing intercepts to vary among years and spatial locations. 

Due to the possibility that plants outside the mapped quadrat could contribute recruits to the focal quadrat or interact with plants in the focal quadrat, we estimated effective cover as a mixture of the observed cover, $C$, in the focal quadrat, $q$, and the mean cover, $\bar{C}$, across the spatial location group, $g$, in which the 
quadrat is located: $C'_{jq,t}=p_j C_{jq,t}+(1-p_j) \bar{C}_{jg,t}$, where $p$ is a mixing fraction between 0 and 1 that was estimated as part of fitting the model.

We ran Markov Chain Monte Carlo (MCMC) simulations in WinBUGS 1.4 \citep{lunn_winbugs_2000} to estimate the recruitment model parameters. Each model was run for 20,000 iterations and two independent chains with different initial values for parameters. We discarded the initial 10,000 samples. Convergence was observed graphically for all parameters, and confirmed by the Brooks-Gelman statistic \citep{brooks_general_1998}. For all three vital rate models, we considered the effect of removal treatments, or other coefficients of interest, to be significant when the 95\% credible interval (or posterior interval) on the estimate of $\chi$ did not overlap zero.  

To incorporate the fitted recruitment function into the IPM, we assumed that individual fecundity increases 
linearly with individual area (equal to $e^u$ for a size-$u$ individual). The recruitment kernel is thus
$R_j(v,u,\vec{n})=c_{0,j}(v)e^{u}\Phi_j$ \citep{adler_coexistence_2010} where $c_{0,j}$ is the
size distribution of new recruits. $\Phi_j$ is calculated from eqn. (\ref{eqn:recrDataModel}) with total cover values computed from the population densities as $C'_{j,t} = \int e^u n_j(u,t) du$. 

\subsection{Modeling local crowding}
We modeled the crowding experienced by a focal genet in each year $t$ as a function of the distance to and size of neighbor genets. In previous work, 
we assumed that the decay of crowding with neighbor distance followed a Gaussian function \citep{chu_large_2015}, but here we use a data-driven 
approach \citep{teller_linking_2016}. We model the crowding experienced by genet $i$ of species $j$ from neighbors of species $m$ as the sum of neighbor 
areas across a set of concentric annuli, $k$, centered at the plant,
\begin{equation}
w_{ijmk,t} = F_{jm}(d_{k})A_{ijmk,t}     
\label{eqn:wik}
\end{equation}
where $F_{jm}$ is the competition kernel (described below) for effects of species $m$ on species $j$, 
$d_{k}$ is the average of the inner and outer radii of annulus $k$, 
and $A_{ijmk,t}$ is the total area of genets of species $m$ in annulus $k$ around the focal genet in year $t$. The total crowding on 
the focal genet exerted by species $m$ is
\begin{equation}
W_{ijm,t}  =\sum_k {w_{ijmk,t}} .
\label{eqn:wijm}
\end{equation} 
Note that $W_{ijj,t}$ gives intraspecific crowding. The $W$'s are then 
included as covariates in the survival (\ref{eqn:survReg}) and growth (\ref{eqn:growReg}) regressions.

We assume that competition kernels $F_{jm}(d)$ are non-negative and decreasing, so that distant plants have less effect 
than close plants. Otherwise, we let the data dictate the shape of the kernel by fitting a spline model 
using the methods of \citet{teller_linking_2016}. The shape of $F_{jm}$ is determined by a set of spline basis coefficients $\vec{b}_{jm}$
and a smoothing parameter $\eta$ that controls the complexity of the fitted kernel. 
Demographic models such as \eqref{eqn:survReg} then have $\gamma$, $\varphi$, $\chi$ , 
$\beta$, $\boldsymbol{\omega}$, $\vec{b}$ and $\eta$ as parameters to be fitted. We implemented this in the statistical computing environment, \texttt{R}, 
by making the spline coefficients and $\eta$ the arguments of an objective function that calculates  each $w_{ijmk,t}$ value using the input spline coefficients, 
calls the model-fitting functions \texttt{lmer} and/or \texttt{glmer} to fit the other parameters in the survival and growth regressions, 
and returns an approximate AIC value and model degrees of freedom ($df$) for survival and growth combined. We used the $\vec{b}$ values at the smoothest 
(smallest $df$) local minimum of AIC as a function of $df$, as in \cite{teller_linking_2016}. This approach assumes that one measure of crowding affects 
survival and growth. In addition, for fitting the kernels we assumed that survival and growth depended only on intraspecific crowding, and thus only fitted the
within-species competition kernels $F_{jj}$. Based on previous work with parametric kernels, \citep{adler_coexistence_2010}, we know that interspecific effects are rarely significantly different from zero, so there is not enough signal to fit interspecific kernels nonparametrically. Therefore, we set all $F_{mj}$ equal to $F_{jj}$, meaning that 
the within-species competition kernel for species $j$ is also used to determine the distance-dependence of species $j$ neighbors on all other species. 
We used data from all historical plots and contemporary control-treatment plots to estimate the competition kernels, which are shown in Fig. \ref{fig:CompKernels}. 

Once we had estimated the competitions kernels, we used them to calculate the $w_{ijmk,t}$ values for each individual in each year, 
and fitted the full survival and growth regressions, which include the interspecific interaction coefficients, $\omega_{jm}$. 
All genets in a quadrat were included in calculating the $w$'s, but plants located within 5 cm of quadrat edges were not used in fitting the regressions. 
We fit the models using the \texttt{R-INLA} package for R \citep{rue_approximate_2009}. We used the same approach when exploring variations 
on the growth regressions, such as the addition of a year-by-treatment interaction or information about the locations of 
individual grasses relative to removed \textit{A. tripartita}. We also compared the species composition of the control and removal plots at both 
the quadrat and neighborhood ($W$) scales.

\subsection{Interspecific covariance in local crowding} 
We explored interspecific covariance in local crowding experienced by individual plants, by regressing the $W$ values exerted by one neighbor species,
 the response variable, against the $W$ values of all other species, the independent variables. Because some $W=0$, we conducted two 
 separate regressions. First, using all $W$'s, we fitted a generalized linear model with a logit link function to evaluate whether the 
 probability that the focal species' $W$ equals 0 is influenced by the value of other species' $W$'s. In this model, the dependent 
 variable is a Bernoulli variate coding for zero versus non-zero value of the focal species' crowding, and the independent 
 variables are the $W$'s for all other species. Second, for the set of records in which the focal species has $W>0$, we 
 performed a linear regression, where the focal species' $W$ is the dependent variable, and the other species' $W$'s are 
 the independent variables. We repeated these regressions for each focal species. Due to large samples size, interspecific $W$ 
 values were often statistically significant predictors of intraspecific. However, they explained very little variance. The 
 maximum reduction in deviance for the generalized linear regressions and $R^2$ for the linear regressions were both less 
 than 8\%. The \texttt{R} code for this analysis is included as ..\texttt{\textbackslash Wdistrib\textbackslash exploreSurvivalWs.r}.
 
\subsection{Correcting for Jensen's inequality in one-step-ahead projections} 
One-year-ahead projections of cover in each quadrat, based on its state in the current year, were made using the regression models for survival, growth and recruitment. 
For each genet present in year $t$, we computed its probability of survival and its expected total cover in year $t+1$, incorporating the random 
year-effects for year $t$ and quadrat group effects. The projected cover in year $t+1$ is the sum over all plants in year $t$ of survival probability times  
projected mean size in year $t+1$ conditional on survival, plus the number of expected recruits multiplied by mean recruit size. 

Projections of mean cover next year have to take account of the fact that our demographic models use log-transformed cover as the size measure, 
and project expected growth and variance in growth on log scale. Random variations in realized growth (on log scale) are modeled as zero-mean Gaussian deviations from the 
regression line, with size-dependent variance. But when we transform those projections back to cover on arithmetic scale, the variability in growth is no longer Gaussian, and 
its mean is no longer zero because of Jensen's inequality. The expected size of an individual next year (conditional on survival) is always larger than the value 
obtained by exponentiating its expected log-transformed size next year, because the exponential function is concave up.  

We therefore developed a correction that takes into account the relationship between plant size and expected variance in growth on log scale, and the 
effect of back-transformation to arithmetic scale, as follows. 

The first step is a data-based estimate of the distribution of random variability in growth for each species. After the growth model for a species was 
fitted, we extracted for each observed size transition in the data set (indexed by $k$) the initial log-transformed size $z_{0,k}$ and the final log-transformed size $z_{1,k}$. 
We used the growth model to compute the expected log-transformed new size $\hat{z}_{1,k}$ and the predicted variance in growth $V_k$, and from those, we 
computed the set of \emph{scaled growth residuals} 
\be
r_k = (z_{1,k} - \hat{z}_{1,k})/\sqrt{V_k} 
\ee 
with $k$ running across all observed size transitions in the data set for that species. We treat the $r_k$ values as a sample from a normalized 
distribution specifying the shape of random growth variability on log scale, and used them to correct one-step-ahead cover 
projections, as follows. 

For individual $j$ present in year $t$ with size $z_j(t)$ let $\hat{z}_j(t+1)$ be its projected mean log-transformed size, and $V_j(t)$ the predicted variance in growth
for that individual, based on its size in year $t$. The projected mean cover on arithmetic scale for individual $j$ in year $t+1$, conditional on survival, 
is then computed as the average of the set of values 
\be
\exp\left(\hat{z}_j(t+1) + \sqrt{V_j(t)}r_k \right).
\ee
The values being exponentiated are possible outcomes of growth with random variability on log scale, their exp's are possible outcomes on arithmetic scale, and
the average of those values is the projected mean cover for the individual, conditional on survival. 

\subsection{Mean field approximation of local crowding for the IPM} 
\label{sec:kernelMethods} 
\citet{adler_coexistence_2010} developed a mean field approximation for local crowding when the
competition kernels are all Gaussian functions, $F_{jm}(d) = e^{-\alpha_{jm} d^2}$. The approximation is explained in 
the online SI to \citet{adler_coexistence_2010} and in section 5.3 of \citet{Ellner2016}. 
Here we explain how that approximation was modified for the IPMs in this paper, which
used fitted nonparametric competition kernels. 

The mean field approximation is based on the observed spatial distribution patterns of the species \citep{adler_coexistence_2010}. 
In both the observed data and IBM simulations, heterospecific individuals were approximately randomly distributed with respect to each other, 
but conspecific individuals displayed a non-random, size-dependent patterns: small genets were randomly distributed, while large genets 
were segregated from each other. The overdispersion of large conspecific genets is incorporated into the IPM 
with a `no-overlap' rule explained below, as in \citep{adler_coexistence_2010}.

For $j \ne m$ (between-species competition), overlap between individuals is allowed. The mean field approximation is 
that from the perspective of any focal plant in species $j$, individuals of species $m$ are distributed at random in space, 
independent of each other and of their size.

Consider the region between the circles of radius $x$ and $x+dx$ centered on a focal genet of species $j$. The area of this annulus
is $2 \pi x \; dx$  to leading order for $dx \approx 0$. A species $m$ genet 
in the annulus puts competitive pressure $F_{jm}(x)$ times its area on
the focal genet. The expected total competitive pressure from all such genets 
is therefore is $F_{jm}(x) 2 \pi x \; dx$ times the expected fractional cover of species $m$ in the annulus 
(fractional cover is the total area of species $m$ genets, as a fraction of the total area). The excepted fractional cover $C_m$ of species $m$
in the annulus equals its fractional cover in the habitat as a whole, because of the assumption of random distribution
spatial distributions. We therefore have $C_m  = \int e^u n_m(u,t) du/A$ where $A$ is the total area of the habitat. 
The total expected competitive pressure on a species-$j$ genet due to species $m$ is then 
\begin{equation}
W_{jm} = \int_0^\infty{C_m F_{jm}(x) 2 \pi x \; dx}  = C_m \left [2 \pi \int_0^{\infty} x F(x) \, dx \right ].
\label{eqn:wbarm}
\end{equation} 
The quantity in square brackets is a constant (that is, it only depends on what the kernel function
is) so it can be computed once and for all for each kernel used in the IPM. The integral is finite because
all fitted kernels fall to zero at a finite distance from the focal plant. 

Our kernel fitting method only uses competition kernel values at the ``mid-ring'' distances
halfway between the inner and outer radii of a series of annuli around each focal
plant, scaled so that the value at the innermost mid-ring distance equals 1. 
In the IPM we defined the kernel at other distances by linear interpolation between values at 
mid-ring distances, except that for the innermost ring a kernel value of 1 was specified at the
outer radius of the ring and at distance $x=0$. 

Now consider within-species competition. We assume that conspecifics cannot overlap. Genet shapes are irregular, but we 
nonetheless implement the no-overlap rule by assuming that a genet of log area $u_i$ is a 
circle of radius $r_i$ where $\pi r_i^2 = e^{u_i}$. The no-overlap rule is then that the centroids of two conspecific individuals 
must be separated by at least the sum of their radii. 

For any one focal genet, the no-overlap restriction on its neighbors' locations affects 
only a negligibly small part of the habitat. The expected cover of individuals in the places
where they can occur (relative to one focal plant) is thus assumed to equal their expected locations
in the habitat as a whole. 
 
Let $C_m(u)$ be the total cover of species $m$ genets of radius $r$ or smaller, 
\begin{equation}
C_m(r) = \int_L^{\log(\pi r^2)}{\! \! \! e^z n_m(z,t) \, dz} .
\label{eqn:cm}
\end{equation}
A focal genet of radius $r$ cannot have any conspecific neighbors centered 
at distances less than $r$. It can have a neighbor centered at distance $x>r$ if that neighbor's
radius is no more than $x-r$. Adding up the expected cover of all such possible neighbors
for a focal genet of radius $r$,    
\begin{equation}
W_{mm}(r) = 2 \pi \int_r^{\infty}F_{mm}(x) x C_m(x-r) \, dx
\label{eqn:wbarmr} 
\end{equation}
This integral is again finite and computable because the kernels $F$ fall to 0 at finite $x$. 

\renewcommand{\refname}{Literature cited}
\bibliographystyle{Ecology}
\bibliography{RemovalRefs}

\clearpage 
\newpage  

\section{Supplementary Tables} 

% supplementary tables
\begin{table}[h]
\caption{Statistical models of year-to-year changes in log(cover) for the four focal species. Values in square brackets show 95\%
 credible intervals.}
\centering
%\begin{center}
\begin{tabular}{l c c c c }
\hline
Species & \textit{A. tripartita} & \textit{H. comata} & \textit{P. secunda} & \textit{P. spicata} \\
\hline
(Intercept)           & $-0.03$          & $0.08$           & $-0.14$          & $0.02$           \\
                      & $[-0.48;\ 0.41]$ & $[-0.14;\ 0.30]$ & $[-0.56;\ 0.27]$ & $[-0.24;\ 0.29]$ \\
TreatmentNo\_grass    & $-0.05$          &                  &                  &                  \\
                      & $[-0.74;\ 0.64]$ &                  &                  &                  \\
TreatmentNo\_shrub    &                  & $0.24$           & $-0.05$          & $0.18$           \\
                      &                  & $[-0.12;\ 0.60]$ & $[-0.35;\ 0.25]$ & $[-0.01;\ 0.38]$ \\
\hline
%AIC                   & 227.74           & 114.65           & 205.41           & 168.17           \\
%BIC                   & 240.01           & 124.69           & 218.18           & 181.40           \\
Log Likelihood        & -108.87          & -52.33           & -97.70           & -79.09           \\
Num. obs.             & 86               & 55               & 95               & 104              \\
Num. groups: quad     & 19               & 11               & 21               & 22               \\
Num. groups: year     & 5                & 5                & 5                & 5                \\
Var: quad (Intercept) & 0.44             & 0.00             & 0.02             & 0.00             \\
Var: year (Intercept) & 0.00             & 0.02             & 0.19             & 0.07             \\
Var: Residual         & 0.52             & 0.36             & 0.38             & 0.24             \\
\hline
\multicolumn{5}{l}{\scriptsize{$^*$ 0 outside the credible interval}}
\end{tabular}
\label{table:coefficients}
%\end{center}
\end{table}

% latex table generated in R 3.2.2 by xtable 1.8-2 package
% Mon Sep 19 08:32:17 2016
\begin{table}[ht]
\centering
\caption{Summary of posterior distributions for fixed effects for the \textit{A. tripartita} survival model. ``logarea" is the effect of plant size, ``Treatment*" is the removal effect, and the ``W.*" coefficients are effects of neighborhood crowding (the $\omega$s in eqn. \ref{eqn:survReg}). ``quant" refers to quantile and ``kld" reports the Kullback-Leibler divergence between the Gaussian and the (simplified) Laplace approximation to the marginal posterior densities. } 
\label{ARTRsurvival}
\begin{tabular}{rrrrrrrr}
  \hline
 & mean & sd & 0.025quant & 0.5quant & 0.975quant & mode & kld \\ 
  \hline
(Intercept) & -0.2613 & 0.1344 & -0.5276 & -0.2605 & 0.0001 & -0.2588 & 0.0000 \\ 
  logarea & 0.7093 & 0.0400 & 0.6315 & 0.7090 & 0.7885 & 0.7085 & 0.0000 \\ 
  TreatmentNo\_grass & -0.9994 & 0.6453 & -2.2156 & -1.0182 & 0.3228 & -1.0565 & 0.0000 \\ 
  W.ARTR & -2.8522 & 0.4187 & -3.6968 & -2.8443 & -2.0513 & -2.8284 & 0.0000 \\ 
  W.HECO & -0.0720 & 0.0615 & -0.1856 & -0.0747 & 0.0565 & -0.0804 & 0.0000 \\ 
  W.POSE & 0.0617 & 0.0895 & -0.1061 & 0.0588 & 0.2459 & 0.0528 & 0.0000 \\ 
  W.PSSP & 0.0387 & 0.0475 & -0.0500 & 0.0370 & 0.1369 & 0.0335 & 0.0000 \\ 
  W.allcov & -0.0065 & 0.0109 & -0.0282 & -0.0065 & 0.0148 & -0.0063 & 0.0000 \\ 
  W.allpts & 0.1293 & 0.1653 & -0.1956 & 0.1294 & 0.4533 & 0.1296 & 0.0000 \\ 
   \hline
\end{tabular}
\end{table}

% latex table generated in R 3.2.2 by xtable 1.8-2 package
% Mon Sep 19 08:33:00 2016
\begin{table}[ht]
\centering
\caption{Summary of posterior distributions for fixed effects for the \textit{H. comata} survival model. See Table \ref{ARTRsurvival} for an explanation of coefficient names
and column headers.} 
\label{HECOsurvival}
\begin{tabular}{rrrrrrrr}
  \hline
 & mean & sd & 0.025quant & 0.5quant & 0.975quant & mode & kld \\ 
  \hline
(Intercept) & 1.4753 & 0.2011 & 1.0764 & 1.4753 & 1.8736 & 1.4753 & 0.0000 \\ 
  logarea & 1.1991 & 0.0747 & 1.0622 & 1.1954 & 1.3561 & 1.1874 & 0.0000 \\ 
  TreatmentNo\_shrub & 0.2915 & 0.3900 & -0.4693 & 0.2898 & 1.0610 & 0.2864 & 0.0000 \\ 
  W.ARTR & -0.0063 & 0.0030 & -0.0121 & -0.0063 & -0.0005 & -0.0063 & 0.0000 \\ 
  W.HECO & -1.6962 & 0.1458 & -1.9894 & -1.6939 & -1.4163 & -1.6891 & 0.0000 \\ 
  W.POSE & 0.0649 & 0.0563 & -0.0442 & 0.0644 & 0.1768 & 0.0634 & 0.0000 \\ 
  W.PSSP & 0.0313 & 0.0360 & -0.0381 & 0.0309 & 0.1031 & 0.0301 & 0.0000 \\ 
  W.allcov & -0.0012 & 0.0062 & -0.0135 & -0.0012 & 0.0110 & -0.0012 & 0.0000 \\ 
  W.allpts & -0.1295 & 0.1004 & -0.3273 & -0.1293 & 0.0670 & -0.1290 & 0.0000 \\ 
   \hline
\end{tabular}
\end{table}

% latex table generated in R 3.2.2 by xtable 1.8-2 package
% Mon Sep 19 08:34:30 2016
\begin{table}[ht]
\centering
\caption{Summary of posterior distributions for fixed effects for the \textit{P. secunda} survival model. See Table \ref{ARTRsurvival} for an explanation of coefficient names
and column headers.} 
\label{POSEsurvival}
\begin{tabular}{rrrrrrrr}
  \hline
 & mean & sd & 0.025quant & 0.5quant & 0.975quant & mode & kld \\ 
  \hline
(Intercept) & 1.4020 & 0.1922 & 1.0176 & 1.4031 & 1.7793 & 1.4050 & 0.0000 \\ 
  logarea & 1.0362 & 0.0645 & 0.9139 & 1.0343 & 1.1696 & 1.0307 & 0.0000 \\ 
  TreatmentNo\_shrub & -0.1604 & 0.1622 & -0.4784 & -0.1606 & 0.1583 & -0.1610 & 0.0000 \\ 
  W.ARTR & 0.0000 & 0.0020 & -0.0038 & 0.0000 & 0.0039 & -0.0000 & 0.0000 \\ 
  W.HECO & -0.0224 & 0.0208 & -0.0631 & -0.0225 & 0.0185 & -0.0227 & 0.0000 \\ 
  W.POSE & -2.0073 & 0.1112 & -2.2290 & -2.0062 & -1.7921 & -2.0038 & 0.0000 \\ 
  W.PSSP & 0.0468 & 0.0238 & 0.0004 & 0.0466 & 0.0939 & 0.0463 & 0.0000 \\ 
  W.allcov & -0.0061 & 0.0035 & -0.0130 & -0.0061 & 0.0008 & -0.0061 & 0.0000 \\ 
  W.allpts & -0.0383 & 0.0567 & -0.1494 & -0.0384 & 0.0730 & -0.0385 & 0.0000 \\ 
   \hline
\end{tabular}
\end{table}

% latex table generated in R 3.2.2 by xtable 1.8-2 package
% Mon Sep 19 08:35:48 2016
\begin{table}[ht]
\centering
\caption{Summary of posterior distributions for fixed effects for the \textit{P. spicata} survival model. See Table \ref{ARTRsurvival} for an explanation of coefficient names
and column headers.} 
\label{PSSPsurvival}
\begin{tabular}{rrrrrrrr}
  \hline
 & mean & sd & 0.025quant & 0.5quant & 0.975quant & mode & kld \\ 
  \hline
(Intercept) & 1.1978 & 0.1639 & 0.8783 & 1.1962 & 1.5257 & 1.1933 & 0.0000 \\ 
  logarea & 1.5138 & 0.0935 & 1.3336 & 1.5122 & 1.7029 & 1.5092 & 0.0000 \\ 
  TreatmentNo\_shrub & -0.2466 & 0.2023 & -0.6454 & -0.2460 & 0.1489 & -0.2449 & 0.0000 \\ 
  W.ARTR & 0.0099 & 0.0022 & 0.0057 & 0.0099 & 0.0142 & 0.0099 & 0.0000 \\ 
  W.HECO & 0.0025 & 0.0289 & -0.0550 & 0.0028 & 0.0586 & 0.0032 & 0.0000 \\ 
  W.POSE & 0.0517 & 0.0389 & -0.0222 & 0.0508 & 0.1308 & 0.0490 & 0.0000 \\ 
  W.PSSP & -1.1267 & 0.0739 & -1.2743 & -1.1258 & -0.9838 & -1.1242 & 0.0000 \\ 
  W.allcov & 0.0131 & 0.0037 & 0.0060 & 0.0131 & 0.0203 & 0.0131 & 0.0000 \\ 
  W.allpts & 0.0756 & 0.0658 & -0.0540 & 0.0757 & 0.2045 & 0.0760 & 0.0000 \\ 
   \hline
\end{tabular}
\end{table}

% latex table generated in R 3.2.2 by xtable 1.8-2 package
% Mon Sep 19 08:36:00 2016
\begin{table}[ht]
\centering
\caption{Summary of posterior distributions for fixed effects for the \textit{A. tripartita} growth model. ``logarea.t0" is the effect of plant size, ``Treatment*" is the removal effect, and the ``W.*" coefficients are effects of neighborhood crowding (the $\omega$s in eqn. \ref{eqn:growReg}. ``quant" refers to quantile and ``kld" reports the Kullback-Leibler divergence between the Gaussian and the (simplified) Laplace approximation to the marginal posterior densities.} 
\label{ARTRgrowth}
\begin{tabular}{rrrrrrrr}
  \hline
 & mean & sd & 0.025quant & 0.5quant & 0.975quant & mode & kld \\ 
  \hline
(Intercept) & 0.7092 & 0.2385 & 0.2396 & 0.7091 & 1.1784 & 0.7090 & 0.0000 \\ 
  logarea.t0 & 0.8689 & 0.0394 & 0.7912 & 0.8689 & 0.9466 & 0.8689 & 0.0000 \\ 
  TreatmentNo\_grass & 0.1385 & 0.1446 & -0.1453 & 0.1385 & 0.4221 & 0.1385 & 0.0000 \\ 
  W.ARTR & -0.1497 & 0.1960 & -0.5345 & -0.1497 & 0.2347 & -0.1497 & 0.0000 \\ 
  W.HECO & 0.0045 & 0.0198 & -0.0343 & 0.0045 & 0.0433 & 0.0045 & 0.0000 \\ 
  W.POSE & -0.0614 & 0.0308 & -0.1218 & -0.0614 & -0.0010 & -0.0614 & 0.0000 \\ 
  W.PSSP & -0.0050 & 0.0158 & -0.0360 & -0.0050 & 0.0260 & -0.0050 & 0.0000 \\ 
  W.allcov & -0.0052 & 0.0041 & -0.0133 & -0.0052 & 0.0029 & -0.0052 & 0.0000 \\ 
  W.allpts & 0.0115 & 0.0536 & -0.0938 & 0.0115 & 0.1167 & 0.0115 & 0.0000 \\ 
   \hline
\end{tabular}
\end{table}

% latex table generated in R 3.2.2 by xtable 1.8-2 package
% Mon Sep 19 08:36:24 2016
\begin{table}[ht]
\centering
\caption{Summary of posterior distributions for fixed effects for the \textit{H. comata} growth model. See Table \ref{ARTRgrowth} for an explanation of coefficient names
and column headers.} 
\label{HECOgrowth}
\begin{tabular}{rrrrrrrr}
  \hline
 & mean & sd & 0.025quant & 0.5quant & 0.975quant & mode & kld \\ 
  \hline
(Intercept) & 0.3937 & 0.0815 & 0.2330 & 0.3938 & 0.5537 & 0.3940 & 0.0000 \\ 
  logarea.t0 & 0.8207 & 0.0213 & 0.7788 & 0.8207 & 0.8626 & 0.8207 & 0.0000 \\ 
  TreatmentNo\_shrub & 0.0691 & 0.1314 & -0.1891 & 0.0691 & 0.3268 & 0.0691 & 0.0000 \\ 
  W.ARTR & -0.0042 & 0.0011 & -0.0063 & -0.0042 & -0.0020 & -0.0042 & 0.0000 \\ 
  W.HECO & -0.2134 & 0.0618 & -0.3348 & -0.2134 & -0.0922 & -0.2135 & 0.0000 \\ 
  W.POSE & 0.0204 & 0.0178 & -0.0146 & 0.0204 & 0.0553 & 0.0204 & 0.0000 \\ 
  W.PSSP & -0.0373 & 0.0131 & -0.0630 & -0.0373 & -0.0117 & -0.0373 & 0.0000 \\ 
  W.allcov & -0.0062 & 0.0027 & -0.0115 & -0.0062 & -0.0010 & -0.0062 & 0.0000 \\ 
  W.allpts & -0.0212 & 0.0416 & -0.1028 & -0.0212 & 0.0604 & -0.0212 & 0.0000 \\ 
   \hline
\end{tabular}
\end{table}

% latex table generated in R 3.2.2 by xtable 1.8-2 package
% Mon Sep 19 08:36:58 2016
\begin{table}[ht]
\centering
\caption{Summary of posterior distributions for fixed effects for the \textit{Poa secunda} growth model. See Table \ref{ARTRgrowth} for an explanation of coefficient names
and column headers.} 
\label{POSEgrowth}
\begin{tabular}{rrrrrrrr}
  \hline
 & mean & sd & 0.025quant & 0.5quant & 0.975quant & mode & kld \\ 
  \hline
(Intercept) & 0.5069 & 0.0625 & 0.3830 & 0.5070 & 0.6299 & 0.5072 & 0.0000 \\ 
  logarea.t0 & 0.6669 & 0.0228 & 0.6215 & 0.6670 & 0.7116 & 0.6673 & 0.0000 \\ 
  TreatmentNo\_shrub & 0.1937 & 0.0632 & 0.0696 & 0.1937 & 0.3176 & 0.1937 & 0.0000 \\ 
  W.ARTR & -0.0004 & 0.0009 & -0.0022 & -0.0004 & 0.0013 & -0.0004 & 0.0000 \\ 
  W.HECO & 0.0110 & 0.0100 & -0.0087 & 0.0110 & 0.0307 & 0.0110 & 0.0000 \\ 
  W.POSE & -0.5753 & 0.0623 & -0.6977 & -0.5753 & -0.4530 & -0.5753 & 0.0000 \\ 
  W.PSSP & -0.0093 & 0.0114 & -0.0316 & -0.0093 & 0.0130 & -0.0093 & 0.0000 \\ 
  W.allcov & -0.0004 & 0.0016 & -0.0035 & -0.0004 & 0.0028 & -0.0004 & 0.0000 \\ 
  W.allpts & -0.0340 & 0.0255 & -0.0842 & -0.0340 & 0.0161 & -0.0340 & 0.0000 \\ 
   \hline
\end{tabular}
\end{table}

% latex table generated in R 3.2.2 by xtable 1.8-2 package
% Mon Sep 19 08:37:48 2016
\begin{table}[ht]
\centering
\caption{Summary of posterior distributions for fixed effects for the \textit{P. spicata} growth model. See Table \ref{ARTRgrowth} for an explanation of coefficient names
and column headers.} 
\label{PSSPgrowth}
\begin{tabular}{rrrrrrrr}
  \hline
 & mean & sd & 0.025quant & 0.5quant & 0.975quant & mode & kld \\ 
  \hline
(Intercept) & 0.3846 & 0.0710 & 0.2425 & 0.3851 & 0.5236 & 0.3858 & 0.0000 \\ 
  logarea.t0 & 0.8259 & 0.0167 & 0.7929 & 0.8259 & 0.8587 & 0.8260 & 0.0000 \\ 
  TreatmentNo\_shrub & 0.2262 & 0.0681 & 0.0924 & 0.2262 & 0.3598 & 0.2262 & 0.0000 \\ 
  W.ARTR & -0.0023 & 0.0008 & -0.0038 & -0.0023 & -0.0007 & -0.0023 & 0.0000 \\ 
  W.HECO & -0.0201 & 0.0111 & -0.0419 & -0.0201 & 0.0017 & -0.0201 & 0.0000 \\ 
  W.POSE & -0.0200 & 0.0137 & -0.0469 & -0.0200 & 0.0069 & -0.0200 & 0.0000 \\ 
  W.PSSP & -0.3845 & 0.0335 & -0.4502 & -0.3845 & -0.3188 & -0.3846 & 0.0000 \\ 
  W.allcov & -0.0048 & 0.0014 & -0.0076 & -0.0048 & -0.0021 & -0.0048 & 0.0000 \\ 
  W.allpts & -0.0510 & 0.0256 & -0.1012 & -0.0510 & -0.0008 & -0.0510 & 0.0000 \\ 
   \hline
\end{tabular}
\end{table}

\begin{table}
\centering
\caption{Summary of fixed effects for the \textit{P. secunda} growth model with treatment*year effects (the ``trtYears*" coefficients). See Table \ref{ARTRgrowth} for an explanation of other coefficient names and column header.} 
\label{table:POSEgrowth-trtYears}
\begin{tabular}{rrrrrrrr}
  \hline
 & mean & sd & 0.025quant & 0.5quant & 0.975quant & mode & kld \\ 
  \hline
(Intercept) & 0.5079 & 0.0626 & 0.3839 & 0.5080 & 0.6310 & 0.5082 & 0.0000 \\ 
  trtYears1 & 0.0910 & 0.1169 & -0.1386 & 0.0910 & 0.3203 & 0.0910 & 0.0000 \\ 
  trtYears2 & 0.1560 & 0.1212 & -0.0819 & 0.1560 & 0.3937 & 0.1560 & 0.0000 \\ 
  trtYears3 & 0.1891 & 0.1461 & -0.0976 & 0.1891 & 0.4757 & 0.1891 & 0.0000 \\ 
  trtYears4 & 0.3125 & 0.1608 & -0.0032 & 0.3125 & 0.6280 & 0.3126 & 0.0000 \\ 
  trtYears5 & 0.3737 & 0.1782 & 0.0239 & 0.3737 & 0.7233 & 0.3737 & 0.0000 \\ 
  logarea.t0 & 0.6663 & 0.0228 & 0.6207 & 0.6664 & 0.7111 & 0.6666 & 0.0000 \\ 
  W.ARTR & -0.0005 & 0.0009 & -0.0022 & -0.0005 & 0.0013 & -0.0005 & 0.0000 \\ 
  W.HECO & 0.0108 & 0.0100 & -0.0089 & 0.0108 & 0.0305 & 0.0108 & 0.0000 \\ 
  W.POSE & -0.5725 & 0.0624 & -0.6950 & -0.5725 & -0.4501 & -0.5725 & 0.0000 \\ 
  W.PSSP & -0.0100 & 0.0114 & -0.0323 & -0.0100 & 0.0123 & -0.0100 & 0.0000 \\ 
  W.allcov & -0.0004 & 0.0016 & -0.0036 & -0.0004 & 0.0028 & -0.0004 & 0.0000 \\ 
  W.allpts & -0.0347 & 0.0256 & -0.0849 & -0.0347 & 0.0155 & -0.0347 & 0.0000 \\ 
   \hline
\end{tabular}
\end{table}


\begin{table}
\centering
\caption{Summary of fixed effects for the \textit{P. spicata} growth model with treatment*year effects (the ``trtYears*" coefficients). See Table \ref{ARTRgrowth} for an explanation of other coefficient names and column headers.} 
\label{table:PSSPgrowth-trtYears}
\begin{tabular}{rrrrrrrr}
  \hline
 & mean & sd & 0.025quant & 0.5quant & 0.975quant & mode & kld \\ 
  \hline
(Intercept) & 0.3948 & 0.0606 & 0.2757 & 0.3946 & 0.5149 & 0.3943 & 0.0000 \\ 
  trtYears1 & 0.3434 & 0.1341 & 0.0800 & 0.3434 & 0.6065 & 0.3434 & 0.0000 \\ 
  trtYears2 & -0.0514 & 0.1391 & -0.3245 & -0.0514 & 0.2215 & -0.0513 & 0.0000 \\ 
  trtYears3 & 0.3547 & 0.1493 & 0.0616 & 0.3547 & 0.6476 & 0.3547 & 0.0000 \\ 
  trtYears4 & 0.5487 & 0.1519 & 0.2505 & 0.5487 & 0.8467 & 0.5487 & 0.0000 \\ 
  trtYears5 & -0.0575 & 0.1598 & -0.3711 & -0.0575 & 0.2560 & -0.0575 & 0.0000 \\ 
  logarea.t0 & 0.8284 & 0.0168 & 0.7951 & 0.8284 & 0.8614 & 0.8285 & 0.0000 \\ 
  W.ARTR & -0.0026 & 0.0008 & -0.0042 & -0.0026 & -0.0011 & -0.0026 & 0.0000 \\ 
  W.HECO & -0.0234 & 0.0109 & -0.0447 & -0.0234 & -0.0020 & -0.0234 & 0.0000 \\ 
  W.POSE & -0.0195 & 0.0137 & -0.0463 & -0.0195 & 0.0073 & -0.0195 & 0.0000 \\ 
  W.PSSP & -0.3590 & 0.0322 & -0.4223 & -0.3590 & -0.2957 & -0.3590 & 0.0000 \\ 
  W.allcov & -0.0053 & 0.0014 & -0.0081 & -0.0053 & -0.0026 & -0.0053 & 0.0000 \\ 
  W.allpts & -0.0567 & 0.0251 & -0.1060 & -0.0567 & -0.0074 & -0.0567 & 0.0000 \\ 
   \hline
\end{tabular}
\end{table}


% latex table generated in R 3.2.2 by xtable 1.8-2 package
% Mon Sep 19 08:37:48 2016
\begin{table}[ht]
\centering
\caption{Summary of posterior distributions for fixed effects for the recruitment model (symbols correspond to Eqns. \ref{eqn:recrDataModel} and \ref{eqn:recrProcessModel}). The indexing on $\omega$ shown here and used in the computer code for the recruitment model gives $\omega$[3,1] as the effect of species 3 on species 1. This is the reverse of the normal convention used in \ref{eqn:recrProcessModel} and shown in \ref{table:alphas}.} 
\label{table:recruitment}
\begin{tabular}{rrrrrrr}
  \hline
 & mean & sd & X2.5. & X97.5. & Rhat & n.eff \\ 
  \hline
 $\gamma$[1] & 0.3341 & 0.7071 & -1.1262 & 1.6391 & 1.0018 &  1100 \\ 
  $\gamma$[2] & 3.4488 & 0.4468 & 2.5010 & 4.3401 & 1.0904 &    25 \\ 
  $\gamma$[3] & 3.2440 & 0.3650 & 2.4840 & 3.9190 & 1.0536 &    34 \\ 
 $\gamma$[4] & 2.9054 & 0.3689 & 2.1820 & 3.6160 & 1.0149 &   110 \\ 
  $\chi$[2,2] & -0.0953 & 0.4093 & -0.8854 & 0.7570 & 1.0010 &  2000 \\ 
 $\chi$[2,3] & -1.2787 & 0.3325 & -1.9370 & -0.6392 & 1.0041 &   420 \\ 
   $\chi$[2,4] & 0.0951 & 0.2617 & -0.4121 & 0.6063 & 1.0023 &   810 \\ 
  $\chi$[3,1] & -1.4366 & 0.8544 & -3.1491 & 0.1132 & 1.0006 &  2000 \\ 
  $\omega$[1,1] & -0.5881 & 0.1018 & -0.7639 & -0.3720 & 1.0197 &    82 \\ 
  $\omega$[1,2] & 0.0635 & 0.0651 & -0.0528 & 0.1997 & 1.0329 &    59 \\ 
  $\omega$[1,3] & 0.0220 & 0.0422 & -0.0610 & 0.1040 & 1.0036 &   920 \\ 
  $\omega$[1,4] & 0.1164 & 0.0456 & 0.0347 & 0.2123 & 1.0197 &    96 \\ 
  $\omega$[2,1] & -0.4800 & 0.1522 & -0.7746 & -0.1906 & 1.0029 &  2000 \\ 
  $\omega$[2,2] & -1.7368 & 0.1297 & -1.9841 & -1.4819 & 1.0006 &  2000 \\ 
  $\omega$[2,3] & 0.0388 & 0.0865 & -0.1287 & 0.2138 & 1.0152 &   100 \\ 
  $\omega$[2,4] & -0.3682 & 0.0926 & -0.5434 & -0.1881 & 1.0005 &  2000 \\ 
  $\omega$[3,1] & -0.6589 & 0.3342 & -1.3131 & -0.0155 & 1.0096 &   280 \\ 
  $\omega$[3,2] & -0.1021 & 0.1981 & -0.4828 & 0.2936 & 1.0089 &   210 \\ 
  $\omega$[3,3] & -1.8943 & 0.1611 & -2.2060 & -1.5850 & 1.0084 &  2000 \\ 
  $\omega$[3,4] & -0.1953 & 0.1508 & -0.4968 & 0.0995 & 1.0074 &  2000 \\ 
  $\omega$[4,1] & -0.1867 & 0.3000 & -0.7608 & 0.4486 & 1.0099 &   190 \\ 
  $\omega$[4,2] & -0.3746 & 0.1896 & -0.7224 & -0.0115 & 1.0453 &    55 \\ 
  $\omega$[4,3] & 0.1131 & 0.1437 & -0.1583 & 0.4170 & 1.0321 &    55 \\ 
  $\omega$[4,4] & -1.7379 & 0.1549 & -2.0400 & -1.4409 & 1.0116 &  1400 \\ 
   $\theta$[1] & 0.6198 & 0.0740 & 0.4835 & 0.7766 & 1.0008 &  2000 \\ 
   $\theta$[2] & 1.1197 & 0.1428 & 0.8666 & 1.4450 & 1.0008 &  2000 \\ 
   $\theta$[3] & 1.1718 & 0.1068 & 0.9685 & 1.3890 & 1.0008 &  2000 \\ 
  $\theta$[4] & 1.0961 & 0.1060 & 0.9009 & 1.3280 & 1.0022 &   890 \\ 
  $p$[1] & 0.8137 & 0.1228 & 0.4813 & 0.9525 & 1.0399 &    60 \\ 
  $p$[2] & 0.9996 & 0.0005 & 0.9984 & 1.0000 & 1.0007 &  2000 \\ 
  $p$[3] & 0.7906 & 0.1288 & 0.4273 & 0.9382 & 1.0176 &  2000 \\ 
  $p$[4] & 0.7502 & 0.1480 & 0.3920 & 0.9453 & 1.0150 &  2000 \\ 
   \hline
   
   \hline
    $\gamma$[1] & 0.2843 & 0.6342 & -1.0900 & 1.4950 & 1.0279 &   810 \\ 
    $\gamma$[2] & 3.2828 & 0.4467 & 2.4429 & 4.1940 & 1.0062 &   260 \\ 
    $\gamma$[3] & 3.2923 & 0.3691 & 2.5680 & 4.0320 & 1.0040 &  1800 \\ 
    $\gamma$[4] & 2.9362 & 0.3811 & 2.1779 & 3.6462 & 1.0116 &   140 \\ 
    $\chi$[2,2] & 0.0405 & 0.4077 & -0.7392 & 0.8147 & 1.0011 &  2000 \\ 
    $\chi$[2,3] & -0.1396 & 0.2658 & -0.6654 & 0.3970 & 1.0015 &  1400 \\ 
    $\chi$[2,4] & 0.4050 & 0.2439 & -0.0664 & 0.8924 & 1.0010 &  2000 \\ 
    $\chi$[3,1] & -1.3691 & 0.6755 & -2.7260 & -0.0685 & 1.0093 &   170 \\ 
      $\omega$[1,1] & -0.5539 & 0.1165 & -0.7517 & -0.2845 & 1.0329 &   240 \\ 
      $\omega$[1,2] & 0.0654 & 0.0581 & -0.0429 & 0.1898 & 1.0025 &   740 \\ 
      $\omega$[1,3] & 0.0043 & 0.0389 & -0.0771 & 0.0813 & 1.0019 &  2000 \\ 
      $\omega$[1,4] & 0.0953 & 0.0410 & 0.0209 & 0.1830 & 1.0120 &   130 \\ 
      $\omega$[2,1] & -0.4522 & 0.1533 & -0.7461 & -0.1556 & 1.0007 &  2000 \\ 
      $\omega$[2,2] & -1.5975 & 0.1175 & -1.8260 & -1.3669 & 1.0033 &  2000 \\ 
      $\omega$[2,3] & -0.0058 & 0.0866 & -0.1806 & 0.1662 & 1.0035 &  2000 \\ 
      $\omega$[2,4] & -0.3047 & 0.0858 & -0.4794 & -0.1381 & 1.0025 &   740 \\ 
      $\omega$[3,1] & -0.5030 & 0.2801 & -1.0880 & 0.0341 & 1.0185 &   120 \\ 
      $\omega$[3,2] & -0.0586 & 0.1571 & -0.3722 & 0.2399 & 1.0053 &   320 \\ 
      $\omega$[3,3] & -1.7052 & 0.1471 & -1.9770 & -1.3980 & 1.0006 &  2000 \\ 
      $\omega$[3,4] & -0.0960 & 0.1304 & -0.3682 & 0.1443 & 1.0005 &  2000 \\ 
      $\omega$[4,1] & -0.1253 & 0.2688 & -0.6245 & 0.3921 & 1.0028 &  2000 \\ 
      $\omega$[4,2] & -0.3056 & 0.1911 & -0.7052 & 0.0560 & 1.0041 &   420 \\ 
      $\omega$[4,3] & 0.0335 & 0.1386 & -0.2499 & 0.2999 & 1.0040 &   440 \\ 
      $\omega$[4,4] & -1.7046 & 0.1597 & -2.0080 & -1.3900 & 1.0101 &   250 \\ 
     $\theta$[1] & 0.6117 & 0.0708 & 0.4855 & 0.7560 & 1.0006 &  2000 \\ 
     $\theta$[2] & 1.1862 & 0.1462 & 0.9392 & 1.5000 & 1.0014 &  2000 \\ 
     $\theta$[3] & 1.1782 & 0.1043 & 0.9872 & 1.4080 & 1.0009 &  2000 \\ 
     $\theta$[4] & 1.2081 & 0.1060 & 1.0090 & 1.4210 & 1.0006 &  2000 \\ 
     $p$[1] & 0.8367 & 0.1095 & 0.5280 & 0.9604 & 1.0235 &  2000 \\ 
     $p$[2] & 0.9995 & 0.0006 & 0.9979 & 1.0000 & 1.0006 &  2000 \\ 
     $p$[3] & 0.8834 & 0.0634 & 0.7124 & 0.9619 & 1.0054 &   410 \\ 
     $p$[4] & 0.8100 & 0.1410 & 0.4269 & 0.9650 & 1.0023 &   830 \\ 
      \hline
\end{tabular}
\end{table}

\newpage 
\begin{table}[ht]
\centering
\caption{Summary of competition coefficients in the demographic models. These are the ``W.abcd'' fixed effects coefficients from
the Tables in this section summarizing the survival, growth and recruitment models for each species.  
In each $4 \times 4$ array of coefficients, the entry in row $i$, column $j$ is the 
coefficient for the effect of species $j$ cover on species $i$. Intraspecific competition is weak relative to intraspecific competition if each
species affects itself more than others, i.e. if the largest (in magnitude) element in each column is the element on the diagonal of the array.
The one exception to this pattern is the effect of POSE on recruitment by ARTR, but the nonconforming
coefficient ($\omega$[3,1] = -0.50 in Table \ref{table:recruitment}) has low precision (sd = 0.28, Table \ref{table:recruitment}.} 
\label{table:alphas}
\begin{tabular}{lrrrr}
       & ARTR &  HECO &  POSE &  PSSP \\
\hline 
\underline{Survival} & & & & \\        
ARTR & -2.8522 & -0.0720 & 0.0617 & 0.0387\\
HECO & -0.0063 & -1.6962 & 0.0649 & 0.0313\\
POSE & 0.0000 & -0.0224 & -2.0073 & 0.0468\\
PSSP & 0.0099 & 0.0025 & 0.0517 & -1.1267\\
\underline{Growth} & & & & \\ 
ARTR & -0.1497 & 0.0045 & -0.0614 & -0.0050\\
HECO & -0.0042 & -0.2134 & 0.0204 & -0.0373\\
POSE & -0.0004 & 0.0110 & -0.5753 & -0.0093\\
PSSP & -0.0023 & -0.0201 & -0.0200 & -0.3845\\
\underline{Recruitment} & & & & \\ 
ARTR & -0.5539 & -0.4522 & -0.5030 & -0.1253\\
HECO &  0.0654 & -1.5975 & -0.0586 & -0.3056\\
POSE & 0.0043 & -0.0058 & -1.7052 & 0.0335\\
PSSP & 0.0953 & -0.3047 & -0.0960 & -1.7046\\
\hline
\end{tabular}
\end{table}

\clearpage

\section{Supplementary Figures} 

 \begin{figure}[h]
 \centering
 \includegraphics[width=0.7\textwidth]{plate}
 \caption{Examples of a grass removal quadrat before treatment in 2011 (A) and four years after treatment in 2015 (B), and an \textit{A. tripartita} removal quadrat before treatment in 2011 (C) and again in 2015. Basal cover of the perennial grasses in (A) was 4.6\% and canopy cover of  \textit{A. tripartita} in (B) was 22.9\%. }
 \label{fig:photos}
 \end{figure}


 \begin{figure}[h]
 \centering
 \includegraphics[width=0.7\textwidth]{PSSP_marginalWARTR}
 \caption{Marginal effects of \textit{A. tripartita} crowding (W) on \textit{P. spicata} growth. Points are residuals from a model that included all 
 covariates from our standard model except for \textit{A. tripartita} crowding. Grey points show (residual) growth of individual plants in control plots, red points show plants in \textit{A. tripartita} removal plots. The blue symbol shows the mean of the red points. Source file: \texttt{growth\textbackslash PSSPgrowth.r}. }
 \label{fig:PSSPresids}
 \end{figure}

 \begin{figure}[h]
 \centering
 \includegraphics[width=0.7\textwidth]{CompKernels}
 \caption{Competition kernels for the four dominant species. Source file: \texttt{Wdistrib\textbackslash exploreSurvivalWs.r}. }
 \label{fig:CompKernels}
 \end{figure}
 
  \begin{figure}[h]
  \centering
  \includegraphics[width=1\textwidth]{climate}
  \caption{Annual precipitation (a) and mean temperature (b) during the period of the experiment, shown against the long-term (1927-2016) means (solid blue and red lines) and 5\% and 95\% quantiles (dashed lines). Source file:  \texttt{climate\textunderscore fig.r}.}
  \label{fig:climate}
  \end{figure}
      
\begin{figure}[h]
\centering
\includegraphics[width=1\textwidth]{growth_residuals_vs_Wremoval}
\caption{The relationship between residuals from the fitted growth model and individual-level pre-treatment crowding exerted by the removal species (all grasses in quadrats where we studied \textit{A. tripartita}, and \textit{A. tripartita} in the quadrats where we studied the grasses). Only individuals present in 2011 in the removal treatment quadrats are included in this analysis. If those individuals survived into later years, we did include their performance in those later years. None of the relationships were statistically significant. Source file:  \texttt{growth\textbackslash growth\textunderscore residuals\textunderscore by\textunderscore individ.R}.}
\label{fig:GrowthResidsIndivid}
\end{figure}
 
  \begin{figure}[tbp]
  \centering
  \includegraphics[width=1\textwidth]{PSSP_W_scatters}
  \caption{Bivariate scatter plots comparing crowding exerted by each neighbor species on \textit{P. spicata}. Each point compares the crowding experienced by one \textit{P. spicata} individual in one year from the species on the x-axis to the species on the y-axis.   Blue symbols show values from controls plots, red symbols show pre-treatment values from \textit{Artemisia} removal plots.``W.allcov'' refers to the aggregated crowding by all shrubs and perennial grasses beyond the focal dominant species, and ``W.allpts'' refers to the aggregated crowding by forb species.  \texttt{Wdistrib\textbackslash exploreSurvivalWs.r}. }
  \label{fig:Wscatters}
  \end{figure}
  
  \begin{figure}[tbp]
 \centering
 \includegraphics[width=1\textwidth]{PSSP_W_byTrt}
 \caption{Distribution of crowding exerted by each neighbor species on \textit{P. spicata} individuals in controls plots (blue) and removal quadrats (red) in 2011, before \textit{A. tripartita}  removals were conducted. Inset bar graphs show the probability that a particular neighbor species was present ($W>0$) in a focal plant's local neighborhood in control (blue) and removal (red) quadrats. ``W.allcov'' refers to the aggregated crowding by all shrubs and perennial grasses beyond the focal dominant species, and ``W.allpts'' refers to the aggregated crowding by forb species.  \texttt{Wdistrib\textbackslash exploreSurvivalWs.r}. }
 \label{fig:W-by-treatment}
 \end{figure}
 

 \begin{figure}[tbp]
 \centering
 \includegraphics[width=1\textwidth]{cover_projections_1step_maxCI}
 \caption{Observed and predicted cover for each of the four modeled species assuming maximum effects of removal. 
  Solid black lines show mean observed cover averaged across control (closed symbols) and removal treatment (open symbols) quadrats. Blue lines and symbols show one-step-ahead predictions from the baseline IBM (no removal treatment coefficients), averaged across quadrats. Red lines and symbols show one-step-ahead predictions from an IBM that includes removal treatment coefficients assigned the outermost value of the 95\% credible intervals. 
  Source file: \texttt{ibm\textbackslash summarize\textunderscore sims1step.r}. }
 \label{fig:IBM1step-maxCI}
 \end{figure}
 
  \begin{figure}[tbp]
  \centering
  \includegraphics[width=1\textwidth]{boxplots-maxCI}
  \caption{Equilibrium cover of the four dominant species simulated by the IPM with maximum removal treatment effects. Boxplots show interannual variation in cover, 
  resulting from random year effects in the demographic models. Gray boxes show cover simulated by a model with \textit{A. tripartita} present and no removal treatment 
  coefficients, blue boxes show results from the same model but with \textit{A. tripartita} cover set to zero (a species removal), red boxes show results from a 
  model with  \textit{A. tripartita} set to zero and (mean) removal treatment coefficients included, and orange boxes show results with removal treatments 
  coefficients assigned the outermost value of the 95\% credible intervals.  Source file: \texttt{ipm\textbackslash IPM-figures.r}.}
  \label{fig:IPMresults-maxCI}
  \end{figure}

\end{document}

